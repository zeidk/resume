%----------------------------------------------------------------------------------------
%	SECTION TITLE
%----------------------------------------------------------------------------------------

\cvsection{Experience}

%----------------------------------------------------------------------------------------
%	SECTION CONTENT
%----------------------------------------------------------------------------------------

\begin{cventries}

%------------------------------------------------


\cventry
{40 h/week} % Job title
{University of Southern California (Senior Research Associate)\newline National Institute of Standards and Technology (Associate)} % Organization
{Los Angeles, CA, USA\newline Gaithersburg, MD, USA} % Location
{October 2022 - Present} % Date(s)
{ % Description(s) of tasks/responsibilities
\begin{cvitems}
\item {Measurement Science for Automated Vehicles.}
    \begin{cvitems2}
    \item {Development of an architecture to characterize the performance of the interaction between modules of an automated vehicle system (Cybersecurity, Artificial Intelligence, Communications, and Perception). The objective of this research is focused on developing a replicable testbed and a suite a test methods and metrics to evaluate system interaction to advance the safe operation of automated vehicles.}
    \end{cvitems2}
\end{cvitems}
}


\cventry
{40 h/week} % Job title
{University of Southern California (Senior Research Associate)\newline National Institute of Standards and Technology (Associate)} % Organization
{Los Angeles, CA, USA\newline Gaithersburg, MD, USA} % Location
{August 2016 - Present} % Date(s)
{ % Description(s) of tasks/responsibilities
\begin{cvitems}
\item {Measurement Science and Simulated Test Methods for Robot Agility.}
    \begin{cvitems2}
    \item {Member of the IEEE P2940\texttrademark\,Standard for Measuring Robot Agility to provide quantitative test methods and metrics for assessing the agility of a robot.}
    \item {Implemented the Agile Robotics for Industrial Automation Competition (ARIAC) in 2020, 2021, 2022, and 2022. Agility challenges and manufacturing tasks were developed in simulation to assess the agility of industrial robots.}
    \end{cvitems2}
    \vspace{7pt}
\item {Machine Learning.}
    \begin{cvitems2}
    \item {Contributed to the development of a methodology to determine the pose of a partially occluded object through the analysis of the latent space of an autoencoder.}
    \item {Contributed to the development of a convolutional autoencoder model to generate a latent space.}
    \item {Developed a methodology to generate synthetic data for a 3D simulated object.}
    \end{cvitems2}
    \vspace{7pt}
\item {Delivery of a knowledge representation and planning infrastructure for kitting operations.}
    \begin{cvitems2}
    \item {Developed a planning infrastructure using the Planning Domain Definition Language (PDDL) to integrate robot capabilities for kitting tasks.}
    \item {Developed a methodology which uses PDDL actions' effects and a graph database to validate the number of objects detected by a vision system.}
    % \item {Developed a tool software to convert ontologies to a graph database.}
    \item {Developed a methodology to update objects in a graph database by comparing their previous poses from the graph database with the new vision data using pre-set tolerance values.}
    % \item {Developed a framework to inspect finished kits using information from a graph database and the vision system.}
    \end{cvitems2}
\end{cvitems}
}



\cventry
{40 h/week} % Job title
{University of Maryland (Assistant Research Scientist)\newline National Institute of Standards and Technology (Associate)} % Organization
{College Park, MD, USA\newline Gaithersburg, MD, USA} % Location
{July 2013 - July 2016} % Date(s)
{ % Description(s) of tasks/responsibilities
\begin{cvitems}
\item {Delivery of a robot agility performance metrics, information models, test methods and protocols to enable manufacturers to easily and rapidly reconfigure and re-task robot systems in assembly operations.}
    \begin{cvitems2}
    \item Integrated  a graph database in pick and place control architecture to perform kitting in real time situations.
    \item {Developed a robot capability model that is intended to be used for: 1) helping manufacturers to characterize the different capabilities their robots contribute to help the end user to select the appropriate robots for the appropriate tasks, 2) selecting backup robots during hardware failures to limit the deterioration of the system’s productivity and the products quality, and 3) limiting robots failures and increasing productivity by providing a tool to manufacturers that outputs a process plan that assigns the best robot to each task needed to accomplish the assembly.}
    % \item {Planning infrastructure, comprised of methods, protocols, and information models to allow for dynamic tasking and re-tasking of robot assembly systems.}
    \end{cvitems2}
\end{cvitems}
}

\cventry
{40 h/week} % Job title
{University of Maryland (Research Associate)\newline National Institute of Standards and Technology (Associate)} % Organization
{College Park, MD, USA\newline Gaithersburg, MD, USA} % Location
{2010 - 2013} % Date(s)
{ % Description(s) of tasks/responsibilities
\begin{cvitems}
\item {Development of the measurement science and standards for planning and
modeling by robots so that they are able to be more quickly re-tasked and
are more flexible and adaptive in manufacturing applications.}
    \begin{cvitems2}
    \item {Developed standard representations for world knowledge and plan knowledge, and the related performance evaluation criteria. }
    \item {Developed techniques to compare planning algorithms that utilize the developed knowledge representations to address next generation robotics for the class of manufacturing problems in the area of component placement.}
    % \item {Examination of input/output standards and performance measures for planning systems.}
    \item {Developed a simulated manufacturing test method that is capable of demonstrating rapid re-tasking and evaluation using our performance measures.}
    \end{cvitems2}
\end{cvitems}
}


\cventry
{40 h/week} % Job title
{National Institute of Standards and Technology (Guest Researcher)} % Organization
{Gaithersburg, MD, USA} % Location
{2005 - 2010} % Date(s)
{ % Description(s) of tasks/responsibilities
\begin{cvitems}
\item {Autonomous driving.}
    \begin{cvitems2}
    \item {Developed the Prediction In Dynamic Environments (PRIDE) framework, a hierarchical multi-resolutional approach for moving object prediction that incorporates multiple prediction algorithms into a single, unifying framework.} 
    \item {Modeled human-like situation awareness capabilities for autonomous ground vehicles to describe the complex set of information that must be maintained in real-time tasks.}
    \item {Integrated a cost-based approach with a Kalman filter approach to generate more accurate predictions of moving obstacles.}
    \item {Developed a fuzzy logic-based approach for identifying objects of interest in the PRIDE framework.}
    \item {Ported the PRIDE framework to manufacturing for loading and unloading materials at docking stations.}
    \end{cvitems2}
    \vspace{10pt}
\item {Urban Search and Rescue.}
    \begin{cvitems2}
    \item {Developed an ontology to analyze the performance of robots for different test methods for Urban Search and Rescue  events.}
    \end{cvitems2}
    \vspace{10pt}
\item {DARPA projects.}
    \begin{cvitems2}
    \item {Contributed to the evaluation and conception of tests to measure the technical performance of devices used in the ASSIST (Advanced Soldier Sensor Information System and Technology) program.}
    \item {Contributed to the development of innovative methods for testing and evaluating hardware and software to securely deploy off-the-shelf smartphones and applications in military field operations for the Transformative Apps (TRANSAPPS) project.}
    \item {Contributed to the evaluation of the performance of technologies used for the TRANSTAC (The Spoken Language Communication and Translation System for Tactical Use) program.}
    \end{cvitems2}
% \item  Development of innovative methods for testing and evaluating hardware and software to securely deploy off-the-shelf smartphones and applications in military field operations for the Transformative Apps (TRANSAPPS) project. TRANSAPPS  is a United States Department of Defense’s “Mobile Apps for the Military” program funded by DARPA. TRANSAPPS provides Soldiers with a number of secure mobile applications that function within a broad range of austere environments.
% \item {Evaluation of the performance of technologies used for the TRANSTAC (The Spoken Language Communication and Translation System for Tactical Use) program. TRANSTAC is a DARPA advanced technology and research program whose goal is to demonstrate capabilities to rapidly develop two-way speech-to-speech translation systems enabling English and foreign language speakers to communicate with one another in real-world tactical situations where an interpreter is unavailable.}
% \item {Construction of test methods and evaluation of the performance of ground, aerial, and aquatic robots in Urban Search and Rescue (US\&R) projects (``Disaster City", College Station, TX).}
\end{cvitems}
}

%------------------------------------------------



%------------------------------------------------

%\cventry
%{Researcher for <Detecting video’s torrents using image similarity algorithms>} % Job title
%{Undergraduate Research, Computer Vision Lab(Prof. Bohyung Han)} % Organization
%{Pohang, S.Korea} % Location
%{Sep. 2012 - Feb. 2013} % Date(s)
%{ % Description(s) of tasks/responsibilities
%\begin{cvitems}
%\item {Researched means of retrieving a corresponding video based on image contents using image similarity algorithm.}
%\item {Implemented prototype that users can obtain torrent magnet links of corresponding video relevant to an image on web site.}
%\end{cvitems}
%}

%------------------------------------------------



%------------------------------------------------

\end{cventries} 